% Sample University of Calgary Thesis
% This file contains CHAPTER ONE

\chapter{Introduction}

\epigraph{The last years of the eighteenth century are broken by a
discontinuity similar to that which destroyed Renaissance thought at
the beginning of the seventeenth.}{Michael Foucault, \emph{The Order
of Things: An Archaeology of the Human Sciences}}

The text above is an \emph{epigraph}. In this typeface, you can have
text in \textbf{boldface}, \emph{italics}, \textbf{\emph{bold
    italics}}, \textsl{slanted}, and \textsc{Small Caps}.

Adjust the commands in the preamble of the main file provided
(that's \verb+sample-thesis.tex+, which you should probably rename to
something more descriptive, like \verb+your-name-thesis.tex+) if you
don't like how this page looks. E.g., you can provide other options for a
different font, or remove the \verb+manuscript+ option for a smaller
font (11~pt) and one-and-a-half linespacing, for a neater, more
compact look. See the \verb+README+ file or the comments in the main
file for details.

The appearance of chapter and section heads, the table of contents,
footnotes, epigraphs (like the one above), and many other book design
elements are controlled by commands provided by the \verb|memoir|
class, which this thesis style uses. Consult the
\href{https://ctan.org/pkg/memoir}{memoir manual} if you want to
change it. However, the standard layout, page numbering, table of
contents, etc., as provided by the \verb|ucalgmthesis| class complies
with the requirements of the
\href{https://grad.ucalgary.ca/current-students/thesis-based-students/thesis/building-thesis}{University
of Calgary thesis formatting guidelines} (July 13, 2018 version).

\section{Literature Review}

\blindtext\pagenote{\blindtext}

\blindtext[2]

\begin{defn}
A group~$G$ is said to be \emph{abelian} (or \emph{commutative}) if
for every $a, b \in G$, $a \cdot b = b \cdot a$.
\end{defn}

\blindtext[2]

\section{Contributions of the Thesis}

\blindtext[3]

\begin{table}
  \begin{center}
  \begin{tabular}{c|c}
    A & B \\
    \hline
    1 & 2
  \end{tabular}
  \end{center}
  \caption{Letters and numbers}
\end{table}

\blindtext[3]
