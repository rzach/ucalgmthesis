% Sample University of Calgary Thesis
% This file contains the FRONT MATTER other than the title page

\chapter{Abstract}

The abstract is a concise and accurate summary of the research
contained in your thesis. It gives the reader a snapshot of your
research by highlighting key points. It includes the problem, method of
study, and general conclusion; relevant key words that will help
people find your research. It must be no longer than 350 words (Note:
hyphenated words or words separated by a slash are counted as two
words). It must be double-spaced or one-half spaced, and not contain
graphs, tables, or illustrations.

\chapter{Preface}

The preface gives a statement of where the information included in
your thesis came from. It gives credit to the authors who informed
your work.  A manuscript-based thesis must include an explanation of
which parts of your thesis were already published and details of the
publication. For example:
\begin{quote}
\textbf{Example 1.} Chapter 1. Portions of the introductory text are
used with permission from Smith et al. (2015) of which I am an author.
Table~1.1 is modified from Supplementary Table~3 in Smith et al.
(2014).

\textbf{Example 2.} Chapter 3 of this thesis has been published as J. Smith and J. Doe, ``Title
of Article''. \emph{Journal Name}, vol. 1, issue 1
\end{quote}
For a traditional thesis requiring ethics approval, include the name
of the board that approved the research project, the title of the
project, and the number of the approval certificate. For example:
\begin{quote}
\textbf{Example.} This thesis is original, unpublished, independent
work by the author, J. Doe.  The experiments reported in Chapters 2--4
were covered by Ethics Certificate number 007, issued by the
University of Calgary Conjoint Health Ethics Board for the project
``Project Title'' on December 15, 2016.
\end{quote}
For a traditional thesis with no ethics approval required, simply
state something along the lines of:
\begin{quote}
\textbf{Example.} This thesis is original, unpublished, independent
work by the author, Jane M. Doe.
\end{quote}

\chapter{Acknowledgments}

The acknowledgments are a place for you to thank/recognize academics
or organizations that supported you (remember you cannot make changes
once your thesis is submitted, so be thoughtful!)  You should include
this when you want to acknowledge people who may or may not be
formally recognized in your thesis elsewhere.  People usually
mentioned in the acknowledgments include your supervisor and
committee, grant support, helpful fellow students, lab mates, or
family members.

Since we're using \LaTeX{} and the \verb|memoir| package, we should
thank Don Knuth, Leslie Lamport, and Peter Wilson.\footnote{These are
the authors of \citet{Knuth1986,Lamport1986,Wilson2016}.}

\dedication{To mom.}

\tableofcontents

% If you have no tables, delete the next line

\listoftables

% If you have no figures, delete the next line

\listoffigures

% Consult Ch 9 of the memoir class manual on how to set up other
% content lists. Note that memoir does not automatically clear the
% page for these. ucalgmthesis fixes this for the default table of
% contents and lists of tables and figures, but not for anything you
% define
