% A Sample Thesis for the University of Calgary
% =============================================

% This is a sample LaTeX document to build a University of Calgary
% graduate thesis according to the guidelines of the Faculty of
% Graduate Studies, available here:
% https://grad.ucalgary.ca/current-students/thesis-based-students/thesis/building-thesis

% To use this sample for your own thesis, rename this file, make any changes
% necessary, then add the content of your thesis t othe included files
% frontmatter.tex, chapter1.tex, etc.

% First, we load the UCalgary Memoir Thesis class ucalgmthesis,
% available at https://github.com/rzach/ucalgmthesis

% By default (without options), this produces a 1-1/2 spaced thesis in
% 11 point font without running heads. See the README file for a
% description of all package options.

% In our sample we give three options: Option utopia sets the thesis in a
% nice font. Option headers produces running heads. Option manuscript
% formats the page in a way suitable for reading and commenting: 12 pt type,
% double spaced, approx. 25 lines per page, with approx. 72 characters
% per line. For filing in the Vault, remove that option to produce a
% more compact thesis with a slightly better layout.

\documentclass[utopia,headers,manuscript]{ucalgmthesis}

% Using LaTeX? Then you're probably using math, and so you want to use
% the AMS math commands and define some theorem environments! But you
% can take these out or use your own favorite theorem package.

\usepackage{amsmath,amsthm}

\newtheorem{thm}{Theorem}
\theoremstyle{definition}
\newtheorem{defn}[thm]{Definition} % please number all of them together!

% microtype makes everything look better

\usepackage{microtype}

% We'll need some colored links, so we load xcolor and hyperref. But
% you can take that out if you don't want links at all.

\usepackage[dvipsnames]{xcolor}

% You can turn off the boxes around links made by hyperref. Then links
% will appear in a different color, and per guidelines, all links must
% be blue or black. For blue links say

\usepackage[colorlinks,allcolors=MidnightBlue]{hyperref} 

% For black links, 
% \usepackage[hidelinks]{hyperref}

% If you prefer hyperref's boxes around links (which don't print), you
% can also change their color. With boxes around links, you probably
% don't want everything in the table of contents to be a link, so we
% only make the page numbers links.
%
% \usepackage[allbordercolors=Periwinkle,linktocpage]{hyperref}

% The table of contents in your PDF reader's sidebar is just titles by
% default, but it's nice to also have chapter and section numbers for
% easy navigation.

\usepackage[numbered]{bookmark}

% For author-year references, you probably want to use natbib with a
% bibliography style appropriate for your discipline; or check out
% latexbib!

\usepackage[round]{natbib}
\bibliographystyle{plainnat}

% The blindtext package produces the ``lorem ipsum''
% texts in this sample and can safely be removed.

\usepackage{blindtext}

% Now we put in the information for the thesis title page.

% Full Name

\author{Jane Mary Doe}

% Full Title

\title{An Important Contribution to the Literature}

% Official name of the degree

\degree{Doctor of Philosophy}

% The name of the graduate program (not the department!)

\prog{Graduate Program in Philosophy}

% The month (for the final version: when you file, not when you defended)

\monthname{May}

% The year

\thesisyear{2019}

% Tell hyperref to put author and title into the PDF metadata

\hypersetup{pdfinfo={Title={\thetitle},Author={\theauthor}}}

% If you want memoir to produce endnotes, turn them on here

\makepagenote

% Often you only want to output a single chapter so you can send it to
% your supervisor. Use includeonly and make sure everything you don't
% always want compiled to PDF is include'd from a separate file. For
% instance, to produce a PDF only of chapter 1, endnotes and
% bibliography, say

% \includeonly{chapter1,backmatter}

% To compile only the title page, which you need when submitting your
% thesis, say

% \includeonly{titlepage}

% and then copy the resulting PDF to a separate file.

\begin{document}

\frontmatter

% titlepage.tex just makes the titlepage; it's in its own file so you
% can typeset it alone using includeonly.

% Sample University of Calgary Thesis
% This file contains the TITLE PAGE

\makethesistitle


% frontmatter.tex contains the abstract, preface, acknowledgments, and
% the commands to produce the table of contents, list of tables, etc.

% Sample University of Calgary Thesis
% This file contains the FRONT MATTER other thna the title page

\chapter{Abstract}

The abstract is a concise and accurate summary of the research
contained in your thesis. It gives the reader a snapshot of your
research by highlighting key points. It includes the problem, method of
study, and general conclusion; relevant key words that will help
people find your research. It must be no longer than 350 words (Note:
hyphenated words or words separated by a slash are counted as two
words). It must be double-spaced or one-half spaced, and not contain
graphs, tables, or illustrations.

\chapter{Preface}

The preface gives a statement of where the information included in
your thesis came from. It gives credit to the authors who informed
your work.  A manuscript-based thesis must include an explanation of
which parts of your thesis were already published and details of the
publication. For example:
\begin{quote}
Portions of the introductory text of Chapter~1 are used with
permission from Smith et al. (2015) of which I am an author. Table~1.1
is modified from Supplementary Table~3 in Smith et
al. (2014). Chapter~2 of this thesis has been published as J.~Smith
and J.~Doe, ``Title of Article''. Journal Name, vol.~1, issue~1.
\end{quote}
For a traditional thesis requiring ethics approval, include the name
of the board that approved the research project, the title of the
project, and the number of the approval certificate. For example:
\begin{quote}
This thesis is original, unpublished, independent work by the author,
J. Doe.  The experiments reported in Chapters 2--4 were covered by
Ethics Certificate number 007, issued by the University of Calgary
Conjoint Health Ethics Board for the project ``Project Title'' on
December 15, 2016.
\end{quote}
For a traditional thesis with no ethics approval required, simply
state something along the lines of:
\begin{quote}
This thesis is original, unpublished, independent work by the author,
Jane M. Doe.
\end{quote}
  
\chapter{Acknowledgments}

The acknowledgments are a place for you to thank/recognize academics
or organizations that supported you (remember you cannot make changes
once your thesis is submitted, so be thoughtful!)  You should include
this when you want to acknowledge people who may or may not be
formally recognized in your thesis elsewhere.  People usually
mentioned in the acknowledgments include your supervisor and
committee, grant support, helpful fellow students, lab mates, or
family members.

Since we're using \LaTeX{} and the \verb|memoir| package, we should
thank Don Knuth, Leslie Lamport, and Peter Wilson.\footnote{These are
the authors of \citet{Knuth1986,Lamport1986,Wilson2016}.}

\dedication{To mom.}

\tableofcontents

% If you have no tables, delete the next line

\listoftables

% If you have no figures, delete the next line

\listoffigures

% Consult Ch 9 of the memoir class manual on how to set up other
% content lists. Note that memoir does not automatically clear the
% page for these. ucalgmthesis fixes this for the default table of
% contents and lists of tables and figures, but not for anything you
% define


% The main matter of the thesis contains the actual content, separated
% into chapters.

\mainmatter

% Sample University of Calgary Thesis
% This file contains CHAPTER ONE

\chapter{Introduction}

\epigraph{The last years of the eighteenth century are broken by a
discontinuity similar to that which destroyed Renaissance thought at
the beginning of the seventeenth.}{Michael Foucault, \emph{The Order
of Things: An Archaeology of the Human Sciences}}

The text above is an \emph{epigraph}. In this typeface, you can have
text in \textbf{boldface}, \emph{italics}, \textbf{\emph{bold
    italics}}, \textsl{slanted}, and \textsc{Small Caps}.

Adjust the commands in the preamble of the main file provided
(that's \verb+sample-thesis.tex+, which you should probably rename to
something more descriptive, like \verb+your-name-thesis.tex+) if you
don't like how this page looks. E.g., you can provide other options for a
different font, or remove the \verb+manuscript+ option for a smaller
font (11~pt) and one-and-a-half linespacing, for a neater, more
compact look. See the \verb+README+ file or the comments in the main
file for details.

The appearance of chapter and section heads, the table of contents,
footnotes, epigraphs (like the one above), and many other book design
elements are controlled by commands provided by the \verb|memoir|
class, which this thesis style uses. Consult the
\href{https://ctan.org/pkg/memoir}{memoir manual} if you want to
change it. However, the standard layout, page numbering, table of
contents, etc., as provided by the \verb|ucalgmthesis| class complies
with the requirements of the
\href{https://grad.ucalgary.ca/current-students/thesis-based-students/thesis/building-thesis}{University
of Calgary thesis formatting guidelines} (July 13, 2018 version).

\section{Literature Review}

\blindtext\pagenote{\blindtext}

\blindtext[2]

\begin{defn}
A group~$G$ is said to be \emph{abelian} (or \emph{commutative}) if
for every $a, b \in G$, $a \cdot b = b \cdot a$.
\end{defn}

\blindtext[2]

\section{Contributions of the Thesis}

\blindtext[3]

\begin{table}
  \begin{center}
  \begin{tabular}{c|c}
    A & B \\
    \hline
    1 & 2
  \end{tabular}
  \end{center}
  \caption{Letters and numbers}
\end{table}

\blindtext[3]


% Sample University of Calgary Thesis
% This file contains CHAPTER TWO

\chapter{Important Results}

\section{The first result}

\begin{thm}[Residue Theorem]
Let $f$ be analytic in the region $G$ except for the isolated 
singularities $a_1,a_2,\dots,a_m$. If $\gamma$ is a closed 
rectifiable curve in $G$ which does not pass through any of the 
points $a_k$ and if $\gamma\approx 0$ in $G$, then
\[
  \frac{1}{2\pi i}\int_\gamma\! f = \sum_{k=1}^m 
  n(\gamma;a_k)\mathop{\mathrm{Res}}(f;a_k)\,.
\]
\end{thm}

\blindtext\footnote{\blindtext}

\begin{figure}
  \[ \circ \to \circ \]
  \caption{Circles and arrows}
\end{figure}

\section{The second result}

\subsection{The first half}
  
\blindtext\pagenote{\blindtext}

\subsection{The second half}

\blindtext


% The back matter includes commands to produce endnotes, index,
% glossary, bibiliography, and the like.

\backmatter

\include{backmatter}

% The appendix contains material that would clutter up the main text,
% such as program code, survey instruments, or interview transcripts.
% Remove it if you don't have an appendix.

\appendix

% Sample University of Calgary Thesis
% This file contains the APPENDIX

% If there is just one appendix, it must be called ``Appendix.'' For
% multiple appendices, use \chapter and a descriptive title,
% e.g., \chapter{Questionnaires}

\chapter*{Appendix}\label{appendix}

% \chapter* doesn't include it in the TOC, so we have to do that by
% hand.  If you have multiple chapters, use \chapter instead and
% remove the following line. 

\addcontentsline{toc}{chapter}{Appendix}

An appendix is a way to include important information that would
otherwise clutter up your thesis. It should be included when there is
additional relevant information that won't fit in the body of your
thesis. Any Appendix must also be mentioned in the body of your thesis
(e.g., ``For a full list of interview questions used, please see the
\hyperref[appendix]{Appendix}''). If your thesis only has one appendix, it
must be titled ``Appendix.'' If your thesis has more than one appendix,
add alphabetized letters, starting with ``Appendix~A.'' The following
are examples of things you might include in appendices:
\begin{enumerate}
  \item Copyright permissions with signatures removed
  \item Additional details of methodology and/or data
  \item Diagrams of equipment that you developed
  \item Digital files and/or artwork digital models
  \item Blank copies of questionnaires or surveys used in your research
\end{enumerate}


\end{document}
